\documentclass[11pt]{scrartcl}
\usepackage[sexy]{{style_files/evan}}

\usepackage{tikz, tkz-euclide}

\usepackage{{style_files/NMC}}
\usepackage{standalone}
\usepackage{import}

\begin{document}
\title{NMC Problem Set \#22} % add # of pset
\date{Jan. 22, 2023} % add date
\maketitle

\section*{Welcome!}

This is a selection of interesting problems derived from curious thoughts, curated so you can nibble on them throughout the week! The point of this document is to introduce you to fun puzzles that require thinking. We recommend you try the ones that you find interesting! Feel free to work on them with others (even us teachers!). Harder problems are marked with chilies (\fullchili), in case you want to challenge yourself.
\newline\newline
Have fun! \textit{Note: New variants on these problems may be released throughout the week. Remember to check back once in a while!}
    
\section{Algebra}
\begin{enumerate}[label=\textbf{A\arabic*}.]
    \item (\halfchili) \textbf{Re-ordering} \newline
    Prove that a certain sequence $(a_n)$ satisfies the following equation for any general choice of $n$,
    \[ 1 + \frac{1}{n-1} + \frac{1}{(n-1)(n-2)} + \dots + \frac{1}{(n-1)!} = a_0 + \frac{a_1}{n} + \frac{a_2}{n^2} + \dots, \]\
    then determine said sequence. Note that the LHS has $n - 1$ terms.    
\end{enumerate}

\newpage
\section{Combinatorics}
\begin{enumerate}[label=\textbf{C\arabic*}.]
    \item \textbf{Josephus Problem} \newline
    $n$ people are standing in a circle, numbered clockwise from $1$ to $n$. Starting from person $1$, we count and eliminate every $k$th person in a clockwise fashion (skipping over people who are already out). The final person remaining wins.

    \begin{enumerate}
        \item In the original problem, there were $n = 41$ people with an elimination parameter of $k = 3$. Who wins the game?
    
        \item (\fullchili) Suppose you are stuck in a given position $j$, but you are given the chance to provide the elimination parameter $k$. Is it possible for you to be the last person standing for all $1 \leq j \leq n$?
    \end{enumerate}
\end{enumerate}

\newpage
\section{Geometry}
\begin{enumerate}[label=\textbf{G\arabic*}.]
    \item (\fullchili) \textbf{Lance's Homework} \newline
    Let $ABC$ be a triangle with $|BC| < |AC|$. Let $M$ be the midpoint of side $AB$, $P$ be the foot of the altitude from $A$, $Q$ be the foot of the altitude from $B$ and $H$ be the orthocenter of $ABC$. Supposing that $QP$ and $AB$ intersect at $T$, prove that $TH$ is perpendicular to $CM$.

    \begin{figure}[h]
        \centering
        \includegraphics[width = 11cm]{weekly/week 22/Diagrams/froglancegeo.tex} % 15cm = aligned?
        \label{fig:froglancegeo}
    \end{figure} 
\end{enumerate}

\newpage
\section{Number Theory}
\begin{enumerate}[label=\textbf{N\arabic*}.]
    \item (\fullchili) \textbf{Prime Sequence with Bertand} \newline
    Bertrand's Postulate asserts that for every positive integer $n$, there is a prime $p$ such that $n < p \leq 2n$. Use this fact to prove that there exists some constant $k \approx 1.25$ where
    \[ \floor{2^k}, \floor{2^{2^k}}, \floor{2^{2^{2^k}}}, \dots \]
    are all prime.

    \item \textbf{Big O for Primes}
    \begin{enumerate}
        \item (\fullchili \hspace{1pt} $\times$ 3) Prove that the average of all primes up to $n$ is less than $n/2$ for $n > 19$. \footnote{tbf, with \href{https://cs.uwaterloo.ca/journals/JIS/VOL18/Axler/axler6.pdf}{this paper} it's $n > 23$, but then again, u can hand verify it works for $n > 19$. Honestly, though, it's only 3 chilis if you don't want to use this paper and want to prove it with variants of the prime number theorem. i found that proof much more fun.}

        \item (\fullchili \hspace{1pt} $\times$ Open) With $p_n$ denoting the $n$th prime, is it true that
        \[ p_{n+1} - p_{n} = O(\log p_n)^2? \]
    \end{enumerate}
\end{enumerate}

\end{document}
