\documentclass[11pt]{scrartcl}
\usepackage[sexy]{{style_files/evan}}

\usepackage{{style_files/NMC}}
\usepackage{standalone}
\usepackage{import}

\begin{document}
\title{NMC Problem Set \#10} % add # of pset
\date{Oct. 23, 2022} % add date
\maketitle

\section*{Welcome!}

This is a selection of interesting problems derived from curious thoughts, curated so you can nibble on them throughout the week! The point of this document is to introduce you to fun puzzles that require thinking. We recommend you try the ones that you find interesting! Feel free to work on them with others (even us teachers!). Harder problems are marked with chilies (\fullchili), in case you want to challenge yourself.
\newline\newline
Have fun! \textit{Note: New variants on these problems may be released throughout the week. Remember to check back once in a while!}
    
\section{Algebra}
\begin{enumerate}[label=\textbf{A\arabic*}.]
    \item \textbf{Powerful Roots} \newline
    Let $r_1, r_2, r_3$ be the roots of $x^3 - x - 1 = 0$. Show that
    \[ r_1^6 + r_2^6 + r_3^6 = 5. \]
    
    \item (\fullchili) \textbf{Polynomial Thing}\footnote{because i just can't think of a better name} \newline
    Let $P$ be a polynomial with positive coefficients. Prove that if
    \[ P\left(\frac{1}{x}\right) \geq \frac{1}{P(x)} \]
    holds for $x = 1$, then it holds for every $x > 0$.
\end{enumerate}

\newpage
\section{Combinatorics}
\begin{enumerate}[label=\textbf{C\arabic*}.]
    \item \textbf{Planar Graph} \newline
    On a flat surface, draw a graph with $n$ vertices. If the edges intersect only at their endpoints/vertices (i.e., no edges pass over each other), then we call the graph \textit{planar}.
    
    \begin{enumerate}
        \item Is the $4$-complete graph ($K_4$) planar?
    
        \item The Euler Characteristic is equal to $V - E + F$, where $V$ represents the number of vertices, $E$ the edges, and $F$ the faces of any polygonal partition of a surface. On a flat surface, the characteristic is $2$. Show that the maximum number of edges we can have in a planar graph with $n$ nodes is $3n - 6$.
        
        \item Provide a combinatorial proof for the above.
        
        \item (\fullchili) Suppose we take this concept of a planar graph but apply it on a torus: if we draw a non-self-intersecting graph on a torus, none of the edges may intersect except at the vertices! What's the maximum number of edges now? Note that the Euler characteristic for a torus is different from that of a surface.
    \end{enumerate}
    
    \item \textbf{Largest Antichain} \newline
    Let $S = \{1, 2, 3, \dots, n\}$. In combinatorics, a collection $F$ of subsets of a given set $S$ is called a \textit{chain} if we can order its elements into $\{s_1, s_2, s_3, \dots\}$ such that
    \[ s_1 \subset s_2 \subset s_3 \subset \dots. \]
    $F$ is called an \textit{antichain} if none of its elements are a subset of one another. For example,
    \[ F = \{\{1, 3, 4, 5\}, \{1, 2, 3, 4\}\} \]
    is an antichain!
    
    \begin{enumerate}
        \item First, let's define the notion of a \textit{power set}, which is commonly written as $\SP(A)$, where $\SP(A)$ contains all the subsets of a set $A$. Let $A$ have $n$ elements. How many elements does $\SP(S)$ have?
        
        \item Let $S^k$ be the family of $k$-element subsets of $S$. For example,
        \[ S^1 = \{\{1\}, \{2\}, \{3\}, \dots, \{n\}\}. \]
        Show that there are $\binom{n}{k}$ sets in $S^k$ and this number is maximized when $k = \lceil {\frac{n}{2}} \rceil$.
        
        \item Show that $S^k$ is an antichain for all choices of $k$, $1 \leq k \leq n$.
        
        Thus, we've concluded that the largest antichain in a set of $n$ elements is at the very least
        \[ \mathrm{min}(\#F) = \binom{n}{\lceil \frac{n}{2} \rceil}. \]
        However - is this the best we can do?
        
        \item (\fullchili) Let's partition the subsets of $S$ into chains, such that no subset of $S$ belongs to $2$ different chains and no chain is fully contained by every element of another chain. Induct to show that such a chain decomposition can contain up to $\binom{n}{\lceil n/2\rceil}$ chains at most.
        
        Conclude by the pigeonhole principle that $\mathrm{max}(\#F) = \mathrm{min}(\#F)$, and we are done.
    \end{enumerate}
\end{enumerate}

\newpage
\section{Geometry}
\begin{enumerate}[label=\textbf{G\arabic*}.]
    \item \textbf{The Hole in the Pizza} \newline
    Lance and Frog are at a restaurant that serves rectangular pizza! However, the chef is very hungry, so he takes a cookie cutter and removes a circle from the pizza, creating a hole. How should the pizza be cut such that we can obtain an even split for both people in one slice?
    
    \begin{figure}[h]
        \centering
        \includegraphics[width = 8cm]{Diagrams/holeinpizza.tex}
        \caption{imagine this as a pizza not a slice of cheese}
        \label{fig:hole_in_pizza}
    \end{figure}
    
    \item \textbf{Unsanitary Environment} \newline
    Suppose we have a $5 \times 5$ grid of tiles. In the beginning, we can pick any $4$ tiles to infect. Every round, tiles that have $2$ infected tiles next to it will subsequently be infected as well. Prove that it's impossible to infect the whole grid after an infinite amount of rounds.
    
\end{enumerate}

\newpage
\section{Number Theory}
\begin{enumerate}[label=\textbf{N\arabic*}.]
    \item \textbf{Predict the Coefficients}
    \begin{enumerate}
        \item Determine the sum of the coefficients of the expansion of $(x + 1)^n$ for any positive integer $n$. What about $(x + a)^n$ for some positive integer $a$? What about the sum of the coefficients of the expansion of $(x^2 + x + 1)^n$ given any positive integer $n$?
        
        \item (\fullchili) Show that the number of odd coefficients in the expansion of $(x+1)^n$ is a power of $2$ for every number $n$.

        \item (\fullchili) Find all prime numbers $p$ for which there exist infinitely many numbers $n$ such that the expansion of $(x+35)^n$ contains no coefficients divisible by $p$.
        
        \item (\fullchili \hspace{1pt} $\times$ 2+) How many coefficients of $(x^2 + x + 1)^n$ have coefficients that aren't divisible by $3$? How many odd coefficients are there?
    \end{enumerate}
\end{enumerate}

\end{document}
